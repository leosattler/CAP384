%%%%%%%%%%%%%%%%%%%%%%%%%%%%%%%%%%%%%%%%%%%%%%%%%%%%%%%%%%%%%%%%%%%%%%%%%%%%%%%

\section*{\large Exercício 7}
\addcontentsline{toc}{chapter}{\protect\numberline{}\large Exercício 7}%

Mostrarei, por indução, a complexidade $\mathcal{O}(n\log_{2}{}n)$ de um algoritmo.

\textbf{Resolução:}

O algoritmo FFT funciona dividindo a soma discreta da DFT de um input de tamanho $n=2^{k}$ em duas somas: uma em $n/2$ termos pares e outra em $n/2$ termos ímpares. Esse procedimento pode ser implementado recursivamente, de modo que a cada passo de divisão pela metade teremos de computar $n$ termos através de somas distintas. 

Um algoritmo com essa característica, que precisa realizar $n$ operações antes, durante e após dividi-las em duas metades, pode ser expresso pela seguinte equação de recorrência ($T_{n}$ denota o valor de uma função $T(n)$ de complexidade de tempo em função do tamanho do input $n$)

\begin{equation*}
 \left\{ \begin{array}{rl} 
T_{n} = 2 T_{\frac{n}{2}} + n,  & n \geq 2  \\
T_{1} = 0.  & 
\end{array}\right.
\end{equation*}

Reescrevendo $T_{n}$:

\begin{align*}
T_{n} = T_{2^{k}} &= 2 T_{2^{k-1}} + 2^{k} \\[10pt]
 \frac{T_{2^{k}}}{2^{k}} &= \frac{2 T_{2^{k-1}} + 2^{k}}{2^{k}} = \frac{2 T_{2^{k-1}}}{2^{k}} + \frac{2^{k}}{2^{k}} \\[10pt]
 \frac{T_{2^{k}}}{2^{k}} &= \frac{T_{2^{k-1}}}{2^{k-1}} + 1 \\[10pt]
 \frac{T_{2^{k}}}{2^{k}} &= \frac{T_{2^{k-2}}}{2^{k-2}} + 1 + 1 \\
 &\vdots \\
 \frac{T_{2^{k}}}{2^{k}} &= \frac{T_{2^{0}}}{2^{0}} + \ldots + 1 + 1 \\[10pt]
 \frac{T_{2^{k}}}{2^{k}} &= 0 + \ldots + 1 + 1 \\[10pt]
 \frac{T_{2^{k}}}{2^{k}} &= k.
\end{align*}

Lembrando que $n=2^{k}$, temos que $k=\log_{2}n$, e a última relação pode ser reescrita:

\begin{align*}
 \frac{T_{2^{k}}}{2^{k}} = \frac{T_{n}}{n} &= \log_{2} n \\
 T_{n} &= n \log_{2} n \tag*{(Q.E.D.)}
\end{align*}

% EXEMPLO PARA ADICIONAR FIGURA
%\begin{figure}[ht!]
	%\caption{Série e histogramas.}
%	\vspace{0mm}	% acrescentar o espaçamento vertical apropriado entre o título e a borda superior da figura
%	\begin{center}
%		\resizebox{15cm}{!}{\includegraphics{Figuras/ex1/Exercicio1_n_64.jpg}}		
%	\end{center}
%	\vspace{-2mm}	% acrescentar o espaçamento vertical apropriado entre a borda inferior da figura e a legenda ou a fonte quando não há legenda (o valor pode ser negativo para subir)
%	\legenda{Figura 1.1: Dez sinais e seus respectivos histogramas para  asérie com $N$ = 64 do grupo noise.}	% legenda - para deixar sem legenda usar comando \legenda{} (nunca deve-se comentar o comando \legenda)
%	\label{ex1_fig1}
%	%\FONTE{}	% fonte consultada (elemento obrigatório, mesmo que seja produção do próprio autor)
%\end{figure}
