%%%%%%%%%%%%%%%%%%%%%%%%%%%%%%%%%%%%%%%%%%%%%%%%%%%%%%%%%%%%%%%%%%%%%%%%%%%%%%%%
% RESUMO %% obrigatório

\begin{resumo}

%% neste arquivo resumo.tex
%% o texto do resumo e as palavras-chave têm que ser em Português para os documentos escritos em Português
%% o texto do resumo e as palavras-chave têm que ser em Inglês para os documentos escritos em Inglês
%% os nomes dos comandos \begin{resumo}, \end{resumo}, \palavraschave e \palavrachave não devem ser alterados

\hypertarget{estilo:resumo}{} %% uso para este Guia

Este relatório trata dos conceitos da aquisição de dados pertinentes à análise de sinais. Em particular, do tempo de observação e da frequência de amostragem de um sinal. Há uma ênfase nos efeitos da frequência de amostragem, ou \textit{sampling}, que leva à introdução dos conceitos de Critério de Nyquist e Frequência de Nyquist. A operação matemática conhecida como convolução também é apresentada. Toda a análise deste relatório é realizada tendo a transformada de Fourier como principal ferramenta, e exemplos são oferecidos de modo a ilustrar os diferentes efeitos da amostragem sobre os resultados desta transformada.

\palavraschave{%
	\palavrachave{Aquisição de dados}%
	\palavrachave{Análise de sinal}%
	\palavrachave{Resolução espectral}%
	\palavrachave{Frequência de Nyquist}%
	\palavrachave{Aliasing}%
}
 
\end{resumo}