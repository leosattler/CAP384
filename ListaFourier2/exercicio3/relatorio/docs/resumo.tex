%%%%%%%%%%%%%%%%%%%%%%%%%%%%%%%%%%%%%%%%%%%%%%%%%%%%%%%%%%%%%%%%%%%%%%%%%%%%%%%%
% RESUMO %% obrigatório

\begin{resumo}

%% neste arquivo resumo.tex
%% o texto do resumo e as palavras-chave têm que ser em Português para os documentos escritos em Português
%% o texto do resumo e as palavras-chave têm que ser em Inglês para os documentos escritos em Inglês
%% os nomes dos comandos \begin{resumo}, \end{resumo}, \palavraschave e \palavrachave não devem ser alterados

\hypertarget{estilo:resumo}{} %% uso para este Guia

O presente relatório estuda o uso de transformadas janeladas de Fourier sobre a função de Cantor. Seis janelas são testadas: Retangular, de Hanning, de Tukey, de Bartlett, de Papoulis e de Hamming. Dois tamanhos de janela são testados sobre três dimensões da função de Cantor. Os espectrogramas de cada caso são produzidos e os resultados discutidos à luz da Análise de Fourier. Utilizou-se \texttt{Python} para as atividades deste relatório. Em particular, a biblioteca \texttt{numpy} para realização da FFT e a biblioteca \texttt{matplotlib} para as visualizações. Os códigos desenvolvidos são oferecidos no repositório deste relatório, bem como as imagens produzidas.


\palavraschave{%
	\palavrachave{Função de Cantor}%
	\palavrachave{Transformada Janelada de Fourier}%
	\palavrachave{Python}%
	\palavrachave{Espectrograma}%
	\palavrachave{Análise tempo-frequência}%
}
 
\end{resumo}