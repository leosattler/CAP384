%%%%%%%%%%%%%%%%%%%%%%%%%%%%%%%%%%%%%%%%%%%%%%%%%%%%%%%%%%%%%%%%%%%%%%%%%%%%%%%

\section*{\large Exercício 1}
\addcontentsline{toc}{chapter}{\protect\numberline{}\large Exercício 1}%

%Bla bla bla

% EXEMPLO PARA ADICIONAR FIGURA
%\begin{figure}[ht!]
	%\caption{Série e histogramas.}
%	\vspace{0mm}	% acrescentar o espaçamento vertical apropriado entre o título e a borda superior da figura
%	\begin{center}
%		\resizebox{15cm}{!}{\includegraphics{Figuras/ex1/Exercicio1_n_64.jpg}}
%	\end{center}
%	\vspace{-2mm}	% acrescentar o espaçamento vertical apropriado entre a borda inferior da figura e a legenda ou a fonte quando não há legenda (o valor pode ser negativo para subir)
%	\legenda{Figura 1.1: Dez sinais e seus respectivos histogramas para  asérie com $N$ = 64 do grupo noise.}	% legenda - para deixar sem legenda usar comando \legenda{} (nunca deve-se comentar o comando \legenda)
%	\label{ex1_fig1}
%	%\FONTE{}	% fonte consultada (elemento obrigatório, mesmo que seja produção do próprio autor)
%\end{figure}

%\subsection*{1.1}
%\addcontentsline{toc}{section}{\protect\numberline{} 1.1}%

Dada a transformada wavelet contínua (CWT) conforme definida abaixo,

\begin{equation*}
\mathcal{W}_{f}^{\psi} = C\int_{-\infty}^{\infty}f(t)\overline{\psi \left(\frac{t-b}{a} \right)}dt, \qquad a>0,
\end{equation*}

tal que

\begin{equation*}
C\int_{-\infty}^{\infty}\left| \psi \left( \frac{t-b}{a} \right) \right| ^{2} dt
\end{equation*}

tenha energia unitária, calcularei o valor de $C$ para as normas $\mathbb{L}^2$ e $\mathbb{L}^1$.

\textbf{Resolução:}

Seja a norma do espaço $\mathbb{L}^2$:

\begin{equation*}
||\psi||_{2} := \left\{\int_{-\infty}^{\infty} |\psi(t)|^{2}dt\right\}^{1/2},
\end{equation*}

onde $\psi \in \mathbb{L}^{2}$. Sendo assim, para qualquer a,b $\in \mathbb{Z}$, $a>0$, temos:

\begin{align*}
||\psi(b/a)||_{2} &= \left\{\int_{-\infty}^{\infty} |\psi\left(\frac{t-b}{a}\right)|^{2}dt\right\}^{1/2} \\
 &= a^{1/2} ||\psi||_{2}.
\end{align*}

Portanto, se uma função $\psi \in \mathbb{L}^{2}$ tem energia unitária, então todas as funções $\psi_{a,b}$ definida por

\begin{equation*}
\psi_{a,b}(t) := a^{-1/2}\psi\left(\frac{t-b}{a}\right), \qquad a,b \in \mathbb{Z},
\end{equation*}

também possuem energia unitária, ou seja,

\begin{equation*}
||\psi_{a,b}||_{2} = ||\psi||_{2}=1, \qquad a,b \in \mathbb{Z}.
\end{equation*}

Sendo assim: 

\begin{equation*}
C := a^{-1/2}.  \tag*{$\blacksquare$}
\end{equation*}

Similarmente para a norma do espaço $\mathbb{L}^{1}$:

\begin{equation*}
||\psi||_{1} := \int_{-\infty}^{\infty} |\psi(t)|^{2}dt,
\end{equation*}

e, sendo assim:

\begin{equation*}
||\psi(b/a)||_{1} = a ||\psi||_{2}.
\end{equation*}

Portanto:

\begin{equation*}
C := a^{-1}.  \tag*{$\blacksquare$}
\end{equation*}





























