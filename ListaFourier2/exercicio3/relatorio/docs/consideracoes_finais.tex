%%%%%%%%%%%%%%%%%%%%%%%%%%%%%%%%%%%%%%%%%%%%%%%%%%%%%%%%%%%%%%%%%%%%%%%%%%%%%%%
\chapter{CONSIDERAÇÕES FINAIS}

A função de Cantor aqui explorada distribui sua energia para mais componentes frequências quanto maior o valor do parâmetro \texttt{seed}. A principal característica da WFT é: quanto maior (menor) a largura da janela, menor (maior) será a resolução temporal da ferramenta e maior (menor) será sua resolução frequencial. Com isso, a WFT requer uma janela de tamanho maior quanto maior o valor de \texttt{seed} para ser capaz de captar com detalhe a contribuição de cada frequência. 

As seis funções janela empregadas foram capazes de analisar conteúdos frequenciais da função de Cantor localmente. Em alguns casos, algumas funções janela foram mais úteis que outras em captar as diferentes frequências da função. Em conjunto, os resultados evidenciam a capacidade da WFT (em especial, do espectrograma) de conferir à análise de Fourier tradicional um componente extra de análise, a saber, o tempo.
 

