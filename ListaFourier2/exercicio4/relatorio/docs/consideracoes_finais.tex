%%%%%%%%%%%%%%%%%%%%%%%%%%%%%%%%%%%%%%%%%%%%%%%%%%%%%%%%%%%%%%%%%%%%%%%%%%%%%%%
\chapter{CONSIDERAÇÕES FINAIS}

Em resumo, a análise de sinais requer um arcabouço teórico robusto desde a aquisição do sinal. Os diversos efeitos da amostragem foram discutidos no presente manuscrito, tendo em vista a análise de Fourier, em particular o conceito de transformada de Fourier e a operação matemática conhecida como convolução. 

Dois fatores foram vistos como cruciais durante a amostragem do sinal: (1) o tamanho da janela de observação $T_{obs}$ afeta a transformada de Fourier de modo que a resolução espectral é $\delta f = \frac{1}{T_{obs}}$, e (2) o \textbf{teorema da amostragem} deve ser satisfeito para evitar aliasing, de modo que a largura de banda do sinal $B$ deve ser pequena o suficiente para satisfazer $B \leq f_{Ny}$ ou $B \leq \frac{f_{samp}}{2}$.