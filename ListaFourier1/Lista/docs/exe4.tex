%%%%%%%%%%%%%%%%%%%%%%%%%%%%%%%%%%%%%%%%%%%%%%%%%%%%%%%%%%%%%%%%%%%%%%%%%%%%%%%

\section*{\large Exercício 4}
\addcontentsline{toc}{chapter}{\protect\numberline{}\large Exercício 4}%

Mostrarei que (comutatividade da convolução):

\begin{align*}
[f \star g] = [g \star f].
\end{align*}

\textbf{Resolução:}

Da definição de convolução, podemos escrever a convolução da função $f$ com a função $g$ como:

\begin{equation*}
(f \star g)(t) = \int_{-\infty}^{+\infty}f(u)g(t-u)d u.
\end{equation*}

Fazendo $\tau = t-u$:

\begin{align*}
(f \star g)(t) &= -\int_{t+\infty}^{t-\infty}f(t - \tau)g(\tau)d \tau \\[10pt]
 &= -\int_{+\infty}^{-\infty}f(t - \tau)g(\tau) d \tau \\[10pt]
 &= \int_{-\infty}^{+\infty}f(t-\tau)g(\tau)d\tau \\[10pt]
 &= (g \star f). \tag*{(Q.E.D.)}
\end{align*}

% EXEMPLO PARA ADICIONAR FIGURA
%\begin{figure}[ht!]
	%\caption{Série e histogramas.}
%	\vspace{0mm}	% acrescentar o espaçamento vertical apropriado entre o título e a borda superior da figura
%	\begin{center}
%		\resizebox{15cm}{!}{\includegraphics{Figuras/ex1/Exercicio1_n_64.jpg}}		
%	\end{center}
%	\vspace{-2mm}	% acrescentar o espaçamento vertical apropriado entre a borda inferior da figura e a legenda ou a fonte quando não há legenda (o valor pode ser negativo para subir)
%	\legenda{Figura 1.1: Dez sinais e seus respectivos histogramas para  asérie com $N$ = 64 do grupo noise.}	% legenda - para deixar sem legenda usar comando \legenda{} (nunca deve-se comentar o comando \legenda)
%	\label{ex1_fig1}
%	%\FONTE{}	% fonte consultada (elemento obrigatório, mesmo que seja produção do próprio autor)
%\end{figure}
