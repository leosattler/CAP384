\documentclass[%
%%% PARA ESCOLHER O ESTILO TIRE O SIMBOLO %(COMENTÁRIO)
%SemVinculoColorido,
%SemFormatacaoCapitulo,
%SemFolhaAprovacao,
%SemImagens,
%CitacaoNumerica, %% o padrão é citação tipo autor-data
%PublicacaoDissOuTese, %% (é também o "default") com ficha catal. e folha de aprovação em branco. Caso tenha lista de símbolos e lista de siglas e abreviaturas retirar os comentários dos arquivos siglas.tex e abreviaturasesiglas.tex. Retirar também os comentários indicados nesse arquivo, nos includes
PublicacaoArtigoOuRelatorio, %% texto sequencial, sem quebra de páginas nem folhas em branco
%PublicacaoProposta, %% igual tese/dissertação, mas sem ficha catal. e fol. de aprov.
%PublicacaoLivro, %% com capítulos
%PublicacaoLivro,SemFormatacaoCapitulo, %% sem capítulos
english,portuguese %% para os documentos em Português com abstract.tex em Inglês
%portuguese,english %% para os documentos em Inglês com abstract.tex em Português
,LogoINPE% comentar essa linha para fazer aparecer o logo do Governo
%,CCBYNC	% as opções de licença são: CCBY, CCBYSA, CCBYND, CCBYNC, CCBYNCSA, CCBYNCND, GPLv3 e INPECopyright
]{tdiinpe}
%]{../../../../../iconet.com.br/banon/2008/03.25.01.19/doc/tdiinpe}

% PARA EXIBIR EM ARIAL TIRAR O COMENTÁRIO DAS DUAS LINHAS SEGUINTES
%\renewcommand{\rmdefault}{phv} % Arial
%\renewcommand{\sfdefault}{phv} % Arial

% PARA PUBLICAÇÕES EM INGLÊS:
% renomear o arquivo: abnt-alf.bst para abnt-alfportuguese.bst
% renomear o arquivo: abnt-alfenglish.bst para abnt-alf.bst


%%%%%%%%%%%%%%%%%%%%%%%%%%%%%%%%%%%%%%%%%%%%%
%%% Pacotes já previamente carregados:      %
%%%%%%%%%%%%%%%%%%%%%%%%%%%%%%%%%%%%%%%%%%%%%%%%%%%%%%%%%%%%%%%%%%%%%%%%
%%% ifthen,calc,graphicx,color,inputenc,babel,hyphenat,array,setspace, %
%%% bigdelim,multirow,supertabular,tabularx,longtable,lastpage,lscape, %
%%% rotate,caption2,amsmath,amssymb,amsthm,subfigure,tocloft,makeidx,  %
%%% eso-pic,calligra,hyperref,ae,fontenc                               %
%%%%%%%%%%%%%%%%%%%%%%%%%%%%%%%%%%%%%%%%%%%%%%%%%%%%%%%%%%%%%%%%%%%%%%%%
%%% insira neste campo, comandos de LaTeX %%%
%%% \usepackage{_exemplo_}
% etc.
\usepackage{xcolor}
\usepackage{rotating}
\usepackage{dsfont}
\usepackage{listings}
\usepackage{booktabs}

\lstset{language=c} 
\lstset{language=C, literate={-}{-}1}

\definecolor{codegreen}{rgb}{0,0.6,0}
\definecolor{codegray}{rgb}{0.3,0.3,0.3}
\definecolor{codepurple}{rgb}{0.58,0,0.82}
\definecolor{backcolour}{rgb}{0.95,0.95,0.92}

\lstdefinestyle{mystyle1}{
    backgroundcolor=\color{backcolour},   
    commentstyle=\color{codegreen},
    keywordstyle=\color{magenta},
    numberstyle=\tiny\color{codegray},
    stringstyle=\color{codepurple},
    basicstyle=\ttfamily\footnotesize,
    breakatwhitespace=false,         
    breaklines=true,                 
    captionpos=b,                    
    keepspaces=true,                 
    numbers=left,                    
    numbersep=5pt,                  
    showspaces=false,                
    showstringspaces=false,
    showtabs=false,                  
    tabsize=2
}

\lstdefinestyle{mystyle2}{
    backgroundcolor=\color{backcolour},   
    commentstyle=\color{codegreen},
    keywordstyle=\color{magenta},
    stringstyle=\color{codepurple},
    basicstyle=\ttfamily\footnotesize,
    breakatwhitespace=false,         
    breaklines=true,                 
    captionpos=b,                    
    keepspaces=true,                                             
    showspaces=false,                
    showstringspaces=false,
    showtabs=false,                  
    tabsize=2
}
%%%%%%%%%%%%%%%%%%%%%%%%%%%%%%%%%%%%%%%%%%%%%

%\watermark{Revisão No. ##} %% use o comando \watermark para identificar a versão de seu documento
%% comente este comando quando for gerar a versão final

%%%%%%%%%%%%%%%%%%%CAPA%%%%%%%%%%%%%%%%%%%%%%%%%%%%%%%%
%\serieinpe{INPE-NNNNN-TDI/NNNN} %% não mais usado

\titulo{Lista Fourier 1 - CAP-384}
\title{Lista Fourier 1 - CAP-384} %% 
\author{Leonardo Sattler Cassara}%\\Nome Completo do Autor2} %% coloque o nome do(s) autor(es)
\descriccao{Lista de Exercícios apresentada aos professores Margarete Domingues e Luciano Magrini como parte da avaliação do curso CAP-384.}
\repositorio{} %% repositório onde está depositado este documento - na omissão, será preenchido pelo SID
\tipoDaPublicacao{}	%% tipo da publicação (NTC, RPQ, PRP, MAN, PUD, TDI, TAE e PRE) na ausência do número de série INPE, caso contrário deixar vazio
\IBI{} %% IBI (exemplo: J8LNKAN8PW/36CT2G2) quando existir, caso contrário o nome do repositório onde está depositado o documento

\date{09 de outubro de 2020}%ano da publicação

%%%%%%%%%%%%%%%%%%%%%%%%%%VERSO DA CAPA%%%%%%%%%%%%%%%%%%%%%%%%%%%%%%%%%%%%%%%%%%%%%%%
\tituloverso{}
\descriccaoverso{}

\descriccaoversoA{}

%%%%%%%%%%%%%%%%%%%FOLHA DE ROSTO

%%%%%%%%%%%%%%%FICHA CATALOGRÁFICA
%% NÃO PREENCHER - SERÁ PREENCHIDO PELO SID

\cutterFICHAC{Cutter}
\autorUltimoNomeFICHAC{Sobrenome, Nomes} %% exemplo: Fuckner, Marcus André
\autorFICHAC {Nome Completo do Autor1; Nome Completo do Autor2} %% Campo opcional (se não usado prevalece \author)
\tituloFICHAC{Titulo da publicação}
\instituicaosigla{INPE}
\instituicaocidade{São José dos Campos}
\paginasFICHAC{\pageref{numeroDePáginasDoPretexto} + \pageref{LastPage}} %% número total de páginas
%\serieinpe{INPE-00000-TDI/0000} %% não mais usado
\palavraschaveFICHAC{1.~Palavra chave. 2.~Palavra chave 3.~Palavra chave. 4.~Palavra chave. 5.~Palavra chave  I.~\mbox{Título}.} %% recomenda-se pelo menos 5 palavras-chaves - \mbox{} é para evitar hifenização 
\numeroCDUFICHAC{000.000} %% número do CDU 

% Nota da ficha (para TD)
\tipoTD{Dissertação ou Tese} % Dissertação ou Tese
\cursoFA{Mestrado ou Doutorado em Nome do Curso}
\instituicaoDefesa{Instituto Nacional de Pesquisas Espaciais}
\anoDefesa{AAAA} % ano de defesa 
\nomeAtributoOrientadorFICHAC{Orientador}	% pode ser: Orientador, Orientadora ou Orientadores
\valorAtributoOrientadorFICHAC{José da Silva} % nome(s) completo(s)

%%%%%%%%%%%%%%%FOLHA DE APROVAÇAO PELA BANCA EXAMINADORA
\tituloFA{\textbf{ATENÇÃO! A FOLHA DE APROVAÇÃO SERÁ INCLUIDA POSTERIORMENTE.}}
%\cursoFA{\textbf{}}
\candidatoOUcandidataFA{}
\dataAprovacaoFA{}
\membroA{}{}{}
\membroB{}{}{}
\membroC{}{}{}
\membroD{}{}{}
\membroE{}{}{}
\membroF{}{}{}
\membroG{}{}{}
\ifpdf

%%%%%%%%%%%%%%NÍVEL DE COMPRESSÃO {0 -- 9}
\pdfcompresslevel 9
\fi
%%% define em 80% a largura das figuras %%%
\newlength{\mylenfig} 
\setlength{\mylenfig}{0.8\textwidth}
%%%%%%%%%%%%%%%%%%%%%%%%%%%%%%%%%%%%%%%%%%%

%%%%%%%%%%%%%%COMANDOS PESSOAIS
\newcommand{\vetor}[1]{\mathit{\mathbf{#1}}} %% faça as modificações pertinentes no arquivo configuracao.tex

\makeindex  %% não alterar, gera INDEX, caso haja algum termo indexado no texto

\begin{document} %% início do documento %% não mexer

%\marcaRegistrada{}	% comando opcional usado para informar abaixo da ficha catalográfica sobre marca registrada
%\marcaRegistrada{Informar aqui sobre marca registrada (a modificação desta linha deve ser feita no arquivo publicacao.tex).}

\maketitle  %% não alterar, gera páginas obrigatórias (folha de rosto, ficha catalográfica e folha de aprovação), automaticamente

%%% Comente as linhas opcionais abaixo caso não as deseje
%\include{./docs/epigrafe} %% Opcional
%\include{./docs/dedicatoria} %% Opcional
%\include{./docs/agradecimentos} %% Opcional
%%%%%%%%%%%%%%%%%%%%%%%%%%%%%%%%%%%%%%%%%%%%%%%%%%%%%%%%%%%%%%%%%%%%%%%%%%%%%%%%
% RESUMO %% obrigatório

\begin{resumo}

%% neste arquivo resumo.tex
%% o texto do resumo e as palavras-chave têm que ser em Português para os documentos escritos em Português
%% o texto do resumo e as palavras-chave têm que ser em Inglês para os documentos escritos em Inglês
%% os nomes dos comandos \begin{resumo}, \end{resumo}, \palavraschave e \palavrachave não devem ser alterados

\hypertarget{estilo:resumo}{} %% uso para este Guia

Este relatório trata dos conceitos da aquisição de dados pertinentes à análise de sinais. Em particular, do tempo de observação e da frequência de amostragem de um sinal. Há uma ênfase nos efeitos da frequência de amostragem, ou \textit{sampling}, que leva à introdução dos conceitos de Critério de Nyquist e Frequência de Nyquist. A operação matemática conhecida como convolução também é apresentada. Toda a análise deste relatório é realizada tendo a transformada de Fourier como principal ferramenta, e exemplos são oferecidos de modo a ilustrar os diferentes efeitos da amostragem sobre os resultados desta transformada.

\palavraschave{%
	\palavrachave{Aquisição de dados}%
	\palavrachave{Análise de sinal}%
	\palavrachave{Resolução espectral}%
	\palavrachave{Frequência de Nyquist}%
	\palavrachave{Aliasing}%
}
 
\end{resumo} %% obrigatório
%\include{./docs/abstract} %% obrigatório

\includeListaFiguras %% obrigatório caso haja mais de 3 figuras, gerado automaticamente
%\includeListaTabelas %% obrigatório caso haja mais de 3 tabelas, gerado automaticamente

%%%%%%%%%%%%%%%%%%%%%%%%%%%%%%%%%%%%%%%%%%%%%%%%%%%%%%%%%%%%%%%%%%%%%%%%%%%%%%%%%
%abreviaturas e siglas  %% opcional, mas recomendado

\begin{abreviaturasesiglas}  %% insira abaixo suas abreviaturas conforme o modelo.

%% sigla (separador: &--&) significado (quebra de linha: \\)
\\
FT   &--& do inglês, \textbf{F}ourier \textbf{T}ransform, ou Transformada de Fourier\\
IFT   &--& do inglês, \textbf{I}nverse \textbf{F}ourier \textbf{T}ransform, ou Transformada Inversa de Fourier\\
FFT    &--&  do inglês, \textbf{F}ast \textbf{F}ourier \textbf{T}ransform, ou Transformada Rápida de Fourier\\
DFT   &--&  do inglês, \textbf{D}iscrete \textbf{F}ourier \textbf{T}ransform, ou Transformada Discreta de Fourier\\


\end{abreviaturasesiglas}
 %% opcional %% altere o arquivo siglaseabreviaturas.tex

%\include{./docs/simbolos} %% opcional %% altere o arquivo simbolos.tex

\includeSumario  %% obrigatório, gerado automaticamente

\newpage
\inicioIntroducao %% não altere este comando

%%%%%%%%%%%%%%%%%%%%%%%%%%%%%%%%%%%%%%%%%%%%%%%%%%%%%%%%%%%%%%%%%%%%%%%%%%%%%%%

\chapter{INTRODUÇÃO}

Amostragem é a redução de um sinal contínuo num sinal discreto e finito. Por sua vez, uma amostra é um valor ou conjunto de valores num determinado ponto do tempo e/ou espaço. Em outras palavras, amostrar um sinal torna-o trabalhável num computador. Como um primeiro passo para o estudo de propriedades de um sinal, seja via análise de Fourier ou qualquer outra ferramenta, este procedimento está sujeito a diversos fatores que afetam profundamente a qualidade da análise. Alguns destes fatores são o tempo total do procedimento de aquisição do sinal e a frequência da amostragem. 

Estes fatores estão diretamente relacionados à robustez da observação: tempo de observação num telescópio e alta taxa de aquisição de dados (e poder de processamento para trabalhar com eles) são custosos. Portanto, quando disponíveis, devem ser explorados ao máximo. O presente manuscrito discute os diferentes efeitos da amostragem sobre a análise de Fourier, em particular sobre a Transformada de Fourier. As melhores condições de amostragem do sinal são detalhadas.

Este relatório está assim organizado: na Seção 2 a operação de convolução, relevante para o assunto deste estudo, é introduzida; na Seção 3 os efeitos da janela de observação são discutidos; na Seção 4 os conceitos de aliasing, critério e frequência de Nyquist são introduzidos; na Seção 5 são oferecidas as considerações finais do autor.

% Paragrafo original (Projeto Fourier)
%O fluxo solar na faixa de 10.7 cm (doravante chamado F10.7) é uma medida da intensidade da emissão do sol na faixa do rádio, mais precisamente em 10.7 cm (ou 2800 MHz). Este índice é um indicador da atividade magnética do Sol, fornecendo informações da atividade solar no ultravioleta e raio-X. Por isso, esse índice é muito relevante em ramos como astrofísica, meteoroglogia e geofísica. Com aplicações em modelagem climática, seu monitoramento é importante para a manutenção dos sistemas de comunicação via satélite \cite{huang2009forecast}. 

%Uma das ferramentas mais usadas para trabalhar com séries temporais deste tipo é a análise espectral, que objetiva representar um sinal como a combinação linear de funções periódicas. Para dados obtidos com um \textit{sampling rate} uniforme, i.e., sob a mesma taxa de registro durante toda a observação, o espectro de potência via FFT (do inglês, Fast Fourier Transform) é o método padrão utilizado. Porém, nem sempre o sinal disponível foi adquirido sob intervalo uniforme. Por exemplo,  o registro da variação do brilho de estrelas via telescópios terrestres está sujeito a diversas interrupções, umas de natureza periódica (rotação e translação terrestre) e outras de natureza não-periódica (mal tempo, problemas do equipamento, etc.). 

%O espectro de potência não é apropriado para dados não uniformes, e uma nova ferramenta se faz necessária para esses casos. O periodograma de Lomb-Scargle \cite{lomb1976least,scargle1982studies} é um algoritmo para detectar e caracterizar a periodicidade de séries temporais com sampling rates não uniformes. Ele utiliza o método de mínimos quadrados para ajustar funções senoidais aos dados \cite{2017arXiv170309824V}. 

%O presente trabalho é um follow-up de \citeonline{Leo}. Os dados F10.7 são manipulados com o fim de simular aquisição não uniforme. Experimentos são efetuados com a simulação de diferentes cenários de sampling rates não uniformes, com a geração do periodograma de Lomb-Scargle utilizando a biblioteca \texttt{astropy}. O presente manuscrito está assim organizado: na Seção 2 a metodologia empregada é introduzida; na Seção 3 os resultados são apresentados com uma breve discussão; na Seção 4 são oferecidas as considerações finais do autor.

% Paragrafo original (Projeto Fourier)
%Em conjunto com os dados de manchas solares, F10.7 é um dos indicadores mais usados para previsão da atividade solar. Por esse motivo, muitos estudos objetivando predição do clima espacial o utilizam como parâmetro de input. Por exemplo, \citeonline{mordvinov1986prediction} utilizou autorregressão multiplicativa para predição mensal dos valores de F10.7. \citeonline{dmitriev1999solar} aplicaram redes neurais para a predição. Por sua vez, \citeonline{zhong2005application} aplicou análise espectral para prever os valores de F10.7. Já \citeonline{bruevich2014study} aplicou análise Wavelet sobre as médias mensais desse dado para análise da série temporal.

% Paragrafo original (Projeto Fourier)
%A análise espectral é um método para representar um sinal como a combinação linear de funções periódicas. Ela faz parte de uma família de técnicas chamadas de Análise de Fourier. No presente trabalho, os dados do índice F10.7 são analisados no contexto da Análise de Fourier. Este manuscrito está assim organizado: na Seção 2 os dados e os tratamentos nele realizados são descritos; na Seção 3 é feita uma recapitulação da Análise de Fourier; na Seção 4 os resultados são apresentados com uma breve discussão; na Seção 5 são oferecidas as considerações finais do autor.
 %% 1o capítulo, começo do texto

%%%%%%%%%%%%%%%%%%%%%%%%%%%%%%%%%%%%%%%%%%%%%%%%%%%%%%%%%%%%%%%%%%%%%%%%%%%%%%%

\chapter{METODOLOGIA}

Bla bla bla.

 %% 2o capítulo
%\clearpage
%%%%%%%%%%%%%%%%%%%%%%%%%%%%%%%%%%%%%%%%%%%%%%%%%%%%%%%%%%%%%%%%%%%%%%%%%%%%%%%


\chapter{RESULTADOS}

Bla bla bla.
 %% 3o capítulo

%\clearpage{}
%%%%%%%%%%%%%%%%%%%%%%%%%%%%%%%%%%%%%%%%%%%%%%%%%%%%%%%%%%%%%%%%%%%%%%%%%%%%%%%
\chapter{CONSIDERAÇÕES FINAIS}

Em resumo, a análise de sinais requer um arcabouço teórico robusto desde a aquisição do sinal. Os diversos efeitos da amostragem foram discutidos no presente manuscrito, tendo em vista a análise de Fourier, em particular o conceito de transformada de Fourier e a operação matemática conhecida como convolução. 

Dois fatores foram vistos como cruciais durante a amostragem do sinal: (1) o tamanho da janela de observação $T_{obs}$ afeta a transformada de Fourier de modo que a resolução espectral é $\delta f = \frac{1}{T_{obs}}$, e (2) o \textbf{teorema da amostragem} deve ser satisfeito para evitar aliasing, de modo que a largura de banda do sinal $B$ deve ser pequena o suficiente para satisfazer $B \leq f_{Ny}$ ou $B \leq \frac{f_{samp}}{2}$. %% 5o capítulo

%\include{./docs/capitulo6} %% 6o capítulo


%% insira quantos capítulos desejar com o seguinte comando:
%\include{_pasta_do_arquivo_/_meu_arquivo_} %%sem a extensão
%%% note que deverá haver um arquivo _meu_arquivo_.tex (com extensão) no diretório _pasta_do_arquivo_

%\include{./docs/conclusao}

\newpage
% Bibliografia %% não alterar %% obrigatório %testebib
\bibliography{./bib/referencia} %% aponte para seu arquivo de bibliografia no formato bibtex (p.ex: referencia.bib)


%\include{./docs/glossario} %% insira os termos do glossário no arquivo glossario.tex %% opcional

%\inicioApendice %% opcional, comente esta linha e a seguintes se não houver apendice(s)
%\include{./docs/apendice1} %% insira apendices tal qual capítulos acima


%\inicioAnexo
%\include{./docs/anexo}
%\include{./docs/anexo1}
%\include{./docs/anexo2}

%\inicioIndice
%\include{./docs/contracapa}
\end{document}