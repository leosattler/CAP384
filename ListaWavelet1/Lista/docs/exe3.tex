%%%%%%%%%%%%%%%%%%%%%%%%%%%%%%%%%%%%%%%%%%%%%%%%%%%%%%%%%%%%%%%%%%%%%%%%%%%%%%%

\section*{\large Exercício 3}
\addcontentsline{toc}{chapter}{\protect\numberline{}\large Exercício 3}%

% EXEMPLO PARA ADICIONAR FIGURA
%\begin{figure}[ht!]
	%\caption{Série e histogramas.}
%	\vspace{0mm}	% acrescentar o espaçamento vertical apropriado entre o título e a borda superior da figura
%	\begin{center}
%		\resizebox{15cm}{!}{\includegraphics{Figuras/ex1/Exercicio1_n_64.jpg}}
%	\end{center}
%	\vspace{-2mm}	% acrescentar o espaçamento vertical apropriado entre a borda inferior da figura e a legenda ou a fonte quando não há legenda (o valor pode ser negativo para subir)
%	\legenda{Figura 1.1: Dez sinais e seus respectivos histogramas para  asérie com $N$ = 64 do grupo noise.}	% legenda - para deixar sem legenda usar comando \legenda{} (nunca deve-se comentar o comando \legenda)
%	\label{ex1_fig1}
%	%\FONTE{}	% fonte consultada (elemento obrigatório, mesmo que seja produção do próprio autor)
%\end{figure}

%====================================================================== 3

Para este exercício, será utilizada a seguinte definição para o produto interno:

\begin{equation*}
\langle f, g \rangle := \int f(x) \overline{g(x)}dx.
\end{equation*}

\textbf{Resolução:}

Considere a relação abaixo,

\begin{equation*}
\int \int \mathfrak{W}_{f}^{\psi}(a,\tau)\overline{\mathfrak{W}_{g}^{\psi}(a,\tau)}d \tau \frac{d a}{a^{2}} = C_{\psi}\langle f, g \rangle.
\end{equation*}

Seja $f$ uma função contínua em $t$. Usando uma função Gaussiana $g_{\alpha}(-t)$ para a função $g$, 

\begin{equation*}
g_{\alpha}(t) := \frac{1}{2\sqrt{\pi a}} e^{-\frac{t^{2}}{4\alpha}},
\end{equation*}

e fazendo $a \rightarrow 0^{+}$, chega-se ao seguinte resultado:

\begin{align*}
f(x) &= \frac{1}{C_{\phi}} \lim_{\alpha \rightarrow 0^{+}} \int_{-\infty}^{\infty}\int_{-\infty}^{\infty}[\mathfrak{W}_{\psi}^{f}(a, \tau)\overline{\langle g_{\alpha}(-t), \psi_{(a, \tau)} \rangle}]\frac{d a}{a^{2}}d\tau \\
&= \frac{1}{C_{\phi}} \int_{-\infty}^{\infty}\int_{-\infty}^{\infty}[\mathfrak{W}_{\psi}^{f}(a, \tau)] \psi_{(a, \tau)}(t)\frac{d a}{a^{2}}d\tau.  \tag*{\text{(Q.E.D.)}}
\end{align*}







