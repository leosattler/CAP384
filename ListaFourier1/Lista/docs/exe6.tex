%%%%%%%%%%%%%%%%%%%%%%%%%%%%%%%%%%%%%%%%%%%%%%%%%%%%%%%%%%%%%%%%%%%%%%%%%%%%%%%

\section*{\large Exercício 6}
\addcontentsline{toc}{chapter}{\protect\numberline{}\large Exercício 6}%

Apresentarei a decomposição da matriz da DFT para um input de tamanho 8 (por exemplo, um sinal representado por um vetor de números reais):

\begin{equation*}
F_{8} = 
\begin{bmatrix}
W_{8}^{0} & W_{8}^{0} & W_{8}^{0} & W_{8}^{0} & W_{8}^{0} & W_{8}^{0} & W_{8}^{0} & W_{8}^{0} \\
W_{8}^{0} & W_{8}^{1} & W_{8}^{2} & W_{8}^{3} & W_{8}^{4} & W_{8}^{5} & W_{8}^{6} & W_{8}^{7} \\
W_{8}^{0} & W_{8}^{2} & W_{8}^{4} & W_{8}^{6} & W_{8}^{8} & W_{8}^{10} & W_{8}^{12} & W_{8}^{14} \\
W_{8}^{0} & W_{8}^{3} & W_{8}^{6} & W_{8}^{9} & W_{8}^{12} & W_{8}^{15} & W_{8}^{18} & W_{8}^{21} \\
W_{8}^{0} & W_{8}^{4} & W_{8}^{8} & W_{8}^{12} & W_{8}^{16} & W_{8}^{20} & W_{8}^{24} & W_{8}^{28} \\
W_{8}^{0} & W_{8}^{5} & W_{8}^{10} & W_{8}^{15} & W_{8}^{20} & W_{8}^{25} & W_{8}^{30} & W_{8}^{35} \\
W_{8}^{0} & W_{8}^{6} & W_{8}^{12} & W_{8}^{18} & W_{8}^{24} & W_{8}^{30} & W_{8}^{36} & W_{8}^{42} \\
W_{8}^{0} & W_{8}^{7} & W_{8}^{14} & W_{8}^{21} & W_{8}^{28} & W_{8}^{35} & W_{8}^{42} & W_{8}^{49}
\end{bmatrix},
\end{equation*}

onde $W_{N}=e^{-2\imath \pi /N}$ é a $N$-ésima raiz de um.


\textbf{Resolução:}

De modo geral, pode-se decompor a matriz da DFT de tamanho $2N \times 2N$ no seguinte produto de matrizes:

\begin{equation*}
F_{2N} = 
\begin{bmatrix}
I_{N} & D_{N}\\
I_{N} & -D_{N} 
\end{bmatrix}
\begin{bmatrix}
F_{N} & \\
 & F_{N} 
\end{bmatrix}
P_{2N},
\end{equation*}

onde $I_{N}$ é a matriz identidade de tamanho $N \times N$, $D_{N}$ é a matriz diagonal com entradas $1, W, \cdots, W^{N-1}$, ou seja, potências consecutivas de $W$:

\begin{equation*}
\raa{.8}
D_{N} = 
\begin{bmatrix}
1 &  &  &  & \\
 & W &  &  & \\
 &  & W^{2} &  & \\
 &  &  & \ddots  & \\
 &  &  &  & W^{N-1}
\end{bmatrix},
\end{equation*}

e $P_{2N}$ é a matriz de permutação de tamanho $2N \times 2N$: 

\begin{equation*}
\raa{.65}
P_{2N} = 
\begin{bmatrix}
1 & 0 & 0 &    0   & \cdots & 0 & 0\\
0 & 0 & 1 &    0   & \cdots & 0 & 0\\
  &   &   & \vdots &        &   &  \\
0 & 0 & 0 &    0   & \cdots & 1 & 0\\
0 & 1 & 0 &    0   & \cdots & 0 & 0\\
0 & 0 & 0 &    1   & \cdots & 0 & 0\\
  &   &   & \vdots &        &   &  \\
0 & 0 & 0 &    0   & \cdots & 0 & 1
\end{bmatrix},
\end{equation*}

que organiza o vetor de input em termos pares e ímpares.

Desse modo, a decomposiçao de $F_{8}$ é:

\begin{equation*}
F_{4} = 
\begin{bmatrix}
I_{4} & D_{4}\\
I_{4} & -D_{4} 
\end{bmatrix}
\begin{bmatrix}
F_{4} & \\
 & F_{4} 
\end{bmatrix}
P_{8},
\end{equation*}

onde $F_{4}$ é:

\begin{equation*}
F_{8} = 
\begin{bmatrix}
W_{4}^{0} & W_{4}^{0} & W_{4}^{0} & W_{4}^{0} \\
W_{4}^{0} & W_{4}^{1} & W_{4}^{2} & W_{4}^{3} \\
W_{4}^{0} & W_{4}^{2} & W_{4}^{4} & W_{4}^{6} \\
W_{4}^{0} & W_{4}^{3} & W_{4}^{6} & W_{4}^{9}
\end{bmatrix}.
\end{equation*}

Portanto, se temos um sinal com oito amostras, representado pelo vetor $\mathbf{f}$, e estamos interessados em calcular sua transformada $\mathbf{\hat{f}}$, podemos realizar a seguinte operação:
\vspace{-2mm}
\begin{align*}
\mathbf{\hat{f}} &= F_{8} \mathbf{f}.
\end{align*} 
Se ao invés do sinal nossos dados forem de um vetor com informações no domínio da frequência, podemos realizar a operação inversa para obter de volta o sinal:
\begin{align*}
\mathbf{\hat{f}} &= F_{8} \mathbf{f} \\
F_{8}^{H} \mathbf{\hat{f}} &= F_{8}^{H} F_{8}  \mathbf{f} = \mathbf{f} \therefore \\
\mathbf{f} &= F_{8}^{H} \mathbf{\hat{f}},
\end{align*} 
onde $F_{8}^{H}$ denota o conjugado Hermitiano (ou conjugado transposto) da matriz $F_{8}$.
%\begin{equation*}
%F_{8} = 
%\begin{bmatrix}
%a & b & c & d & a & b & c & d \\
%d & e & f & d & a & b & c & d \\
%g & h & i & d & a & b & c & d \\
%a & b & c & d & a & b & c & d \\
%d & e & f & d & a & b & c & d \\
%g & h & i & d & a & b & c & d \\
%a & b & c & d & a & b & c & d \\
%d & e & f & d & a & b & c & d 
%\end{bmatrix}
%\end{equation*}

% EXEMPLO PARA ADICIONAR FIGURA
%\begin{figure}[ht!]
	%\caption{Série e histogramas.}
%	\vspace{0mm}	% acrescentar o espaçamento vertical apropriado entre o título e a borda superior da figura
%	\begin{center}
%		\resizebox{15cm}{!}{\includegraphics{Figuras/ex1/Exercicio1_n_64.jpg}}		
%	\end{center}
%	\vspace{-2mm}	% acrescentar o espaçamento vertical apropriado entre a borda inferior da figura e a legenda ou a fonte quando não há legenda (o valor pode ser negativo para subir)
%	\legenda{Figura 1.1: Dez sinais e seus respectivos histogramas para  asérie com $N$ = 64 do grupo noise.}	% legenda - para deixar sem legenda usar comando \legenda{} (nunca deve-se comentar o comando \legenda)
%	\label{ex1_fig1}
%	%\FONTE{}	% fonte consultada (elemento obrigatório, mesmo que seja produção do próprio autor)
%\end{figure}
