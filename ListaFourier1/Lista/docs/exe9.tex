%%%%%%%%%%%%%%%%%%%%%%%%%%%%%%%%%%%%%%%%%%%%%%%%%%%%%%%%%%%%%%%%%%%%%%%%%%%%%%%

\section*{\large Exercício 9}
\addcontentsline{toc}{chapter}{\protect\numberline{}\large Exercício 9}%

Entregue em 13/10/2020.

% EXEMPLO PARA ADICIONAR FIGURA
%\begin{figure}[ht!]
	%\caption{Série e histogramas.}
%	\vspace{0mm}	% acrescentar o espaçamento vertical apropriado entre o título e a borda superior da figura
%	\begin{center}
%		\resizebox{15cm}{!}{\includegraphics{Figuras/ex1/Exercicio1_n_64.jpg}}		
%	\end{center}
%	\vspace{-2mm}	% acrescentar o espaçamento vertical apropriado entre a borda inferior da figura e a legenda ou a fonte quando não há legenda (o valor pode ser negativo para subir)
%	\legenda{Figura 1.1: Dez sinais e seus respectivos histogramas para  asérie com $N$ = 64 do grupo noise.}	% legenda - para deixar sem legenda usar comando \legenda{} (nunca deve-se comentar o comando \legenda)
%	\label{ex1_fig1}
%	%\FONTE{}	% fonte consultada (elemento obrigatório, mesmo que seja produção do próprio autor)
%\end{figure}
